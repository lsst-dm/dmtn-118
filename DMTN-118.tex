\documentclass[DM,authoryear,toc]{lsstdoc}
% lsstdoc documentation: https://lsst-texmf.lsst.io/lsstdoc.html

\input{meta}

% Package imports go here.

% Local commands go here.

% To add a short-form title:
% \title[Short title]{Title}
\title{Review of Timeseries Features}

% Optional subtitle
% \setDocSubtitle{A subtitle}

\author{%
Eric Bellm
}

\setDocRef{DMTN-118}
\setDocUpstreamLocation{\url{https://github.com/lsst-dm/dmtn-118}}

\date{\vcsDate}

% Optional: name of the document's curator
% \setDocCurator{The Curator of this Document}

\setDocAbstract{%
The DPDD allocates space for pre-computed timeseries features, and a sample set is baselined in LDM-151. However, other features have been developed. This technote reviews the relevant literature, grouping related features where possible, and discusses potential concerns.
}

% Change history defined here.
% Order: oldest first.
% Fields: VERSION, DATE, DESCRIPTION, OWNER NAME.
% See LPM-51 for version number policy.
\setDocChangeRecord{%
  \addtohist{1}{YYYY-MM-DD}{Unreleased.}{Eric Bellm}
}

\begin{document}

% Create the title page.
\maketitle

% ADD CONTENT HERE
% You can also use the \input command to include several content files.

\appendix
% Include all the relevant bib files.
% https://lsst-texmf.lsst.io/lsstdoc.html#bibliographies
\section{References} \label{sec:bib}
\bibliography{local,lsst,lsst-dm,refs_ads,refs,books}

% Make sure lsst-texmf/bin/generateAcronyms.py is in your path
\section{Acronyms} \label{sec:acronyms}
\addtocounter{table}{-1}
\begin{longtable}{p{0.145\textwidth}p{0.8\textwidth}}\hline
\textbf{Acronym} & \textbf{Description}  \\\hline

ADQL & Astronomical Data Query Language \\\hline
AGN & active galactic nuclei \\\hline
ANTARES & Arizona-NOA Temporal Analysis and Response to Events System \\\hline
AP & Alert Production \\\hline
API & Application Programming Interface \\\hline
B & Byte (8 bit) \\\hline
DM & Data Management \\\hline
DMTN & DM Technical Note \\\hline
DPDD & Data Product Definition Document \\\hline
DRP & Data Release Production \\\hline
LDM & LSST Data Management (Document Handle) \\\hline
LSE & LSST Systems Engineering (Document Handle) \\\hline
LSST & Legacy Survey of Space and Time (formerly Large Synoptic Survey Telescope) \\\hline
PPDB & Prompt Products DataBase \\\hline
PSF & Point Spread Function \\\hline
RFC & Request For Comment \\\hline
SN & SuperNovae \\\hline
TAI & International Atomic Time \\\hline
TAP & Table Access Protocol \\\hline
VLT & Very Large Telescope (ESO) \\\hline
VO & Virtual Observatory \\\hline
ZTF & Zwicky Transient Facility \\\hline
\end{longtable}


\end{document}
